\documentclass[10pt]{beamer}

\usetheme[progressbar=frametitle]{metropolis}
\usepackage{appendixnumberbeamer}
\usepackage[numbers,sort&compress]{natbib}
\usepackage[utf8]{inputenc}
\usepackage[american]{babel}
\bibliographystyle{plainnat}
\usepackage{booktabs}
\usepackage[scale=2]{ccicons}
\usepackage{amsmath, amsthm}
\usepackage{xspace}
\usepackage{enumerate}
\newcommand{\themename}{\textbf{\textsc{metropolis}}\xspace}

\definecolor{darkgreen}{rgb}{0,0.5,0}
\definecolor{navy blue}{RGB}{0, 0, 99}
\definecolor{rock blue}{RGB}{156, 170, 198}
\definecolor{refs}{gray}{0.7}

\setbeamercolor{attention box}{ %
  fg=navy blue, %
  bg=refs
}

% Coisas invisiveis ficam meio transparentes
\setbeamercovered{transparent=40} 

% % Desabilita os menuzinhos chatos
% \setbeamertemplate{navigation symbols}{} 

% % Faz com que as tabelas e figuras sejam numeradas
% \setbeamertemplate{caption}[numbered]

% \AtBeginSubsection[]{
% \begin{frame}<beamer>
% \frametitle{Resumo da aula}
% \tableofcontents[currentsection,currentsubsection]
% \end{frame}
% }
\newcommand{\pP}{\mathcal{P}}
\newcommand{\D}{\mathcal{D}}
\newcommand{\R}{\mathbb{R}}
\newcommand{\Na}{\mathbb{N}}
\newcommand{\FS}{\mathcal{T}}
\newcommand{\Ac}{\mathcal{A}}
\newcommand{\Eq}{\mathcal{E}}
\newcommand{\Ineq}{\mathcal{I}}

\DeclareMathOperator{\posto}{posto}

% Triple norms
\newcommand{\vertiii}[1]{ %
  { %
    \left\vert\kern-0.25ex\left\vert\kern-0.25ex\left\vert #1
        \right\vert\kern-0.25ex\right\vert\kern-0.25ex\right\vert
  }}

\DeclareMathOperator{\tr}{tr}

\title{Modelagem computacional de problemas de otimização}
\subtitle{Dia 1}

\author{Francisco N. C. Sobral \and Giovana Melo dos Santos \and Joaquim
  Gabriel Martins} \date{XXXIII SEMAT 2023} \institute{DMA -
  Universidade Estadual de Maringá}

\begin{document}

\maketitle

\begin{frame}{Resumo}

  \tableofcontents
  
\end{frame}

\section{Motivação}

\begin{frame}{Modelagem}

  \begin{itemize}
  \item A \alert{modelagem matemática} de um problema está associada à
    \alert{escrever matematicamente} um problema qualquer

  \item Em otimização, geralmente estamos interessados em olhar para
    \alert{conjunto de pontos} e escolher algum
  \end{itemize}
  
\end{frame}

\begin{frame}{Problema do churrasco}

  \begin{itemize}
  \item Uma pessoa vai fazer um churrasco para \alert<2>{$N$ pessoas}
    
  \item Vai comprar contra filé (R\$ 50 o kg) e costela (R\$ 20 o
    kg). Para cada pessoa, \alert<2>{calcula-se $500$g de carne se for contra
    filé ou $700$g se for costela}
    
  \item Deve haver um balanço entre a gordura e a proteína:
    \alert<2>{no máximo 20g de gordura e no mínimo 50g de
      proteína} por pessoa em média

  \item Balanço nutricional fictício por 100g

    \begin{center}
      \begin{tabular}{lll}
        \hline
        & Gordura & Proteína \\ \hline
        Costela & 5g & 5g \\ \hline
        Contra filé & 10g & 2g \\ \hline
      \end{tabular}
    \end{center}
  \end{itemize}
  
\end{frame}

\begin{frame}{Um problema de otimização}

  \begin{onlyenv}<1>
    Um problema de otimização geral é dado por
    \[
      \begin{array}{ll}
        \mbox{minimizar} & f(x) \\
        \mbox{sujeito a} & x \in \Omega
      \end{array}
    \]
    com $f: \R^n \to \R$ e $\Omega \in \R^n$
  \end{onlyenv}
  
  \begin{onlyenv}<2-3>
    Um problema de otimização linear\only<3>{\alert{ binária}} é dado por
    \[
      \begin{array}{ll}
        \mbox{minimizar} & c_1 x_1 + \dots + c_n x_n \\
        \mbox{sujeito a} & a_{i1} x_1 + \dots + a_{in} x_n \le b_i, \quad i \in I_1 \\
        & a_{i1} x_1 + \dots + a_{in} x_n = b_i, \quad i \in E \\
        & a_{i1} x_1 + \dots + a_{in} x_n \ge b_i, \quad i \in I_2 \\
        & x_i \alt<3>{\alert{\in \{0, 1\}}}{\ge 0}
      \end{array}
    \]
  \end{onlyenv}

\end{frame}

\begin{frame}{Hipóteses}

  Se um problema é modelado através de um problema de programação
  linear, então assumimos
  
  \begin{itemize}
  \item Proporcionalidade

    A contribuição de uma variável é proporcional ao seu valor
    
  \item Aditividade

    O valor da função objetivo total é a soma das contribuições
    individuais, o mesmo para as restrições
    
  \item Divisibilidade

    O valor de uma variável pode ser fracionado o quanto se necessite
    (valores inteiros não são obrigatórios)
    
  \item Dados determinísticos

    As constantes são bem definidas

  \end{itemize}
  
\end{frame}

\begin{frame}{Problema do churrasco}

  \begin{itemize}
  \item Uma pessoa vai fazer um churrasco para $N$ pessoas
    \alert{minimizando o custo financeiro}
    
  \item Vai comprar contra filé (R\$ 50 o kg) e costela (R\$ 20 o
    kg). Para cada pessoa, calcula-se $500$g de carne se for contra
    filé ou $700$g se for costela
    
  \item Deve haver um balanço entre a gordura e a proteína: no máximo
    20g de gordura e no mínimo 50g de proteína por pessoa em média

  \item Balanço nutricional fictício por 100g

    \begin{center}
      \begin{tabular}{lll}
        \hline
        & Gordura & Proteína \\ \hline
        Costela & 5g & 5g \\ \hline
        Contra filé & 10g & 2g \\ \hline
      \end{tabular}
    \end{center}
  \end{itemize}
  
\end{frame}

\section{Usando JuMP}

\begin{frame}{JuMP}

  \begin{itemize}
  \item JuMP -- Julia for Mathematical Programming

  \item Site: \url{https://jump.dev}
  \end{itemize}
  
\end{frame}

\begin{frame}[fragile]{Primeiro modelo}

  \begin{itemize}
  \item<only@1> Considere
  \[
    \begin{array}{ll}
      \mbox{Minimizar} & 12 x + 20 y \\
      \mbox{Sujeita a} & 6x + 8y \ge 100 \\
                       & 7x + 12y \ge 120 \\
                       & x \ge 0 \\
                       & y \in [0, 3]
    \end{array}
  \]
  
  Retirado de {\scriptsize
    \url{https://jump.dev/JuMP.jl/stable/tutorials/getting_started/getting_started_with_JuMP/#An-example}}
    
  \item<only@2> Assumo que JuMP e HiGHS já estão instalados (Nosso ambiente já faz isso)

  \item<only@3> Usando os pacotes
\begin{verbatim}
using JuMP, HiGHS
\end{verbatim}

  \item<only@4> Criando o modelo para ser resolvido pelo software HiGHS
\begin{verbatim}
modelo = Model(HiGHS.Optimizer)
\end{verbatim}

  \item<only@5> Criando as variáveis (com suas restrições)
\begin{verbatim}
@variable(modelo, x >= 0)
@variable(modelo, 0 <= y <= 3)
\end{verbatim}

  \item<only@6> Adicionando as restrições
\begin{verbatim}
@constraint(modelo, c1, 6x + 8y >= 100)
@constraint(modelo, c2, 7x + 12y >= 120)
\end{verbatim}

  \item<only@7> Função objetivo
\begin{verbatim}
@objective(modelo, Min, 12x + 20y)
\end{verbatim}

  \item<only@8> Resolvendo
\begin{verbatim}
optimize!(modelo)
\end{verbatim}

  \item<only@9> Mostra alguns resultados
\begin{verbatim}
@show termination_status(modelo)
@show primal_status(modelo)
@show dual_status(modelo)
@show objective_value(modelo)
@show value(x)
@show value(y)
\end{verbatim}
    
  \end{itemize}
  
\end{frame}

\begin{frame}[fragile]{Criando variáveis}

  \begin{center}
\begin{verbatim}
@variable(modelo, x)
@variable(modelo, x[i = 1:10] >= 0)
@variable(modelo, x[1:5, 1:3]

u = [1, 3, 4, 5, 6]
@variable(modelo, x[i = 1:5] <= u[i])
\end{verbatim}
  \end{center}
  
\end{frame}

\begin{frame}[fragile]{Criando restrições}

  \begin{center}
\begin{verbatim}
@constraint(modelo, 2 x[1] - x[3] == -10)

a = [-10, 0, 3]
@constraint(modelo, sum(x[j] for j = 1:3) == -10)

A = [0 1 1;
     1 2 0]
b = [8, 10];
@constraint(modelo, [i = 1:2],
            sum(A[i, j] * x[j] for j = 1:3) <= b[i])
\end{verbatim}
  \end{center}
  
\end{frame}

\begin{frame}[fragile]{Notação matricial}

\begin{verbatim}
A = [ 1 1 9  5
      3 5 0  8
      2 0 6 13 ]

b = [7, 3, 5]
c = [1, 3, 5, 2]

modelo = Model(HiGHS.Optimizer)
@variable(modelo, x[1:4] >= 0)
@constraint(modelo, A * x .== b)
@objective(modelo, Min, c' * x)

optimize!(modelo)
\end{verbatim}
  
\end{frame}

\section{Alguns exemplos}

\begin{frame}{O problema da dieta (animal)}

  \begin{itemize}
  \item<only@1> Uma empresa agrícola fabrica ração para aves
  \item<only@1> Diversos ingredientes são misturados (milho, soja,
    \dots) de forma a atingir a quantidade mínima de determinados
    nutrientes (cálcio, proteína, \dots)

  \item<only@2> Suponha que tenhamos \alert{$n$ ingredientes}
  \item<only@2> Suponha que tenhamos \alert{$m$ nutrientes}
  \item<only@2> Seja $x_j$ a quantidade utilizada para o ingrediente
    $j$
  \item<only@2> \textcolor{red}{Qual o problema de otimização que
      vamos modelar?}

  \item<only@3> Minimizar o custo de produção
    \[
      \sum_{j = 1}^n c_j x_j,
    \]
    onde $c_j$ é o custo para adquirir \alert{uma unidade} do produto
    $j$
    
  \item<only@4> Produzir uma quantidade $b$ do produto final
    \[
      \sum_{j = 1}^n x_j = b
    \]
  \item<only@5> Satisfazer limites \alert{mínimos} e \alert{máximos}
    de cada nutriente
    \[
      b \bar l_i \le \sum_{j = 1}^n a_{ij} x_j \le b \bar u_i, \quad i =
      1, \dots, m
    \]
    onde
    \begin{itemize}
    \item $0 \le \bar l_i \le \bar u_i$ são os limites para \alert{uma
        unidade} do nutriente $i$
    \item $a_{ij}$ é a quantidade de nutriente $i$ que há \alert{em
        uma unidade} do ingrediente $j$
    \end{itemize}

  \item<only@6> Há um limite superior para a quantidade do ingrediente
    $j$ que pode ser comprada
    \[
      x_j \le u_j, \quad j = 1, \dots, n
    \]
  \end{itemize}

  \begin{onlyenv}<7>
    \[ 
      \begin{array}{ll}
        \mbox{Minimizar} & \displaystyle \sum_{j = 1}^n c_j x_j \\
        \mbox{Sujeita a} & \displaystyle \sum_{j = 1}^n x_j = b \\[0.4cm]
                         & \displaystyle b \bar l_i \le \sum_{j = 1}^n a_{ij} x_j \le b \bar u_i, i = 1, \dots, m \\[0.3cm]
                         & 0 \le x_j \le u_j, i = 1, \dots, n
      \end{array}
    \]
  \end{onlyenv}
  
\end{frame}

\begin{frame}
  \frametitle{Problema do Sudoku}

  \begin{onlyenv}<1>
    \begin{itemize}
    \item Queremos resolver um Sudoku com dimensão $4 \times 4$
    \item Temos um quadrado com 16 posições, 4 linhas, 4 colunas e 4
      blocos $2 \times 2$
    \item São dados os valores conhecidos $v_{ij} \in \mathcal{C}$
    \item Em cada linha, coluna ou bloco deve existir um e apenas uma
      ocorrência do número $n \in \{1 \dots, 4\}$
    \end{itemize}
  \end{onlyenv}

  \begin{onlyenv}<2-4>
    \begin{itemize}
    \item<only@2> Utilizaremos variáveis binárias
    \item<only@2> Seja $x_{ijk}$ indicando 1 se a posição $(i, j)$ possui o
      valor $k$, e 0 caso contrário, para $i, j, k \in \{1, \dots, 4\}$
    \item<only@3> Somente um valor valor por linha $\ell$
      \[
        \sum_{j = 1}^4 \sum_{k = 1}^4 x_{\alert{\ell}, j, k} = 1
      \]
    \item<only@3> Somente um valor por coluna $c$
      \[
        \sum_{i = 1}^4 \sum_{k = 1}^4 x_{i, \alert{c}, k} = 1
      \]
    \item<only@3> Somente um valor por bloco $(b_i, b_j)$, $b_i, b_j \in \{1, 2, 3\}$
      \[
        \alert{\sum_{i = 2(b_i - 1)}^{2b_i} \sum_{j = 2(b_j - 1)}^{2b_j}} \sum_{k = 1}^4 x_{i, j, k} = 1
      \]
    \item<only@4> $x \in \{0, 1\}$

    \item<only@4> Lugares já preenchidos, para cada
      $v_{ij} \in \mathcal{C}$
      \[
        x_{ij\ v_{ij}} = 1
      \]
    \end{itemize}
  \end{onlyenv}

\end{frame}

\begin{frame}{Problema do financiamento}

  \begin{itemize}
  \item<only@1> A prefeitura de Maringá tem um grande projeto para ser
    executado em 4 anos
  \item<only@1> Os gastos previstos (em milhões) para cada ano são
    \begin{center}
      \begin{tabular}{cccc}
        1 & 2 & 3 & 4 \\\hline
        2 & 4 & 8 & 5
      \end{tabular}
    \end{center}
  \item<only@2> A prefeitura irá emitir títulos de dívida (ou seja,
    empréstimos) para realizar a obra que pagam juros
    \textbf{simples} com \alert{previsão de taxas} dada por
    \begin{center}
      \begin{tabular}{cccc}
        1 & 2 & 3 & 4 \\\hline
        7\% & 6\% & 6,5\% & 7,5\%
      \end{tabular}
    \end{center}
  \item<only@2> Os títulos começarão a cobrar juros \alert{após} o fim
    da obra (quarto ano) e serão quitados após 20 anos

  \item<only@3> Além disso, a prefeitura pode colocar dinheiro em uma
    \alert{aplicação}, para usar em anos posteriores, que lhe rendem
    juros anuais previstos como
    \begin{center}
      \begin{tabular}{cccc}
        1 & 2 & 3 & 4 \\\hline
        6\% & 5,5\% & 4,5\% & -
      \end{tabular}
    \end{center}
    (No quarto ano, não faz sentido guardar dinheiro)

  \item<only@4> Seja $x_j$ o valor obtido pela venda de títulos de
    dívida no ano $j$, $j = 1, \dots, 4$
  \item<only@4> Seja $y_j$ a quantia colocada na aplicação no ano $j$,
    $j = 1, 2, 3$

  \item<only@5> Desejamos minimizar a quantidade de juros paga
    \[
      20 \cdot 0,07 x_1 + 20 \cdot 0,06 x_2 + 20 \cdot 0,065 x_3 + 20
      \cdot 0,075 x_4
    \]
  \item<only@6> Em cada ano, o valor usado no projeto ou vem dos
    títulos vendidos ou vem das aplicações de anos anteriores

  \item<only@6> Uma parte do valor obtido nos títulos pode ser
    guardada em aplicação para uso futuro
    \[
      \begin{aligned}
        x_1 \alert{- y_1} & = 2 \\
        \alert{1,06 y_1} + x_2 - y_2 & = 4 \\
        1,055 y_2 + x_3 - y_3 & = 8 \\
        1,045 y_3 + x_4 & = 5
      \end{aligned}
    \]
  \end{itemize}

  \begin{onlyenv}<7>
    \[
      \begin{array}{ll}
        \mbox{Minimizar} & 20 \cdot 0,07 x_1 + 20 \cdot 0,06 x_2 + 20 \cdot 0,065 x_3 + 20\cdot 0,075 x_4 \\
        \mbox{Sujeita a} &         x_1 - y_1 = 2 \\
                         & 1,06 y_1 + x_2 - y_2  = 4 \\
                         & 1,055 y_2 + x_3 - y_3  = 8 \\
                         & 1,045 y_3 + x_4 = 5 \\
                         & x_j \ge 0,\quad j = 1, \dots, 4 \\
                         & y_j \ge 0,\quad j = 1, 2, 3
      \end{array}
    \]
  \end{onlyenv}
  
\end{frame}

\section{O grande problema}

\begin{frame}
  \frametitle{O problema}

  \begin{itemize}
  \item Os professores do DMA têm direito a 3 meses consecutivos de
    afastamento a cada 5 anos
  \item A cada mês, no máximo 20\% dos professores pode estar
    afastados
  \item Uma vez recebido, os professores têm um prazo de 5 anos para
    usufruir senão perdem o afastamento
  \item Atualmente, cada professor $p$ tem até o prazo $t_p$ para
    usufruir de seu afastamento. Após esse prazo, ele perde o atual e
    ganha direito a outro
  \item Queremos planejar os afastamentos para o horizonte dos
    próximos 5 anos
  \end{itemize}
  
\end{frame}

\end{document}